\documentclass[11pt,authoryear]{article}
\pdfoutput = 1
\usepackage{fullpage,amsthm,amsmath,natbib,algorithm,algorithmic,enumitem,afterpage,amssymb,setspace,graphicx,amsfonts,array,verbatim,commath,mathrsfs}
\usepackage[dvipsnames]{xcolor}
\usepackage{tikz}
\usetikzlibrary{arrows,matrix,positioning, fit, shapes.geometric}
\usepackage[hang,flushmargin]{footmisc}
\usepackage[colorlinks=TRUE,citecolor=Blue,linkcolor=BrickRed,urlcolor=PineGreen]{hyperref}
\newtheorem{lemma}{Lemma}
\newtheorem{theorem}{Theorem}
\newtheorem{proposition}{Proposition}
\newtheorem{corollary}{Corollary}
\newtheorem{claim}{Claim}
\newtheorem{definition}{Definition}
\newtheorem{exmp}{Example}[section]
\allowdisplaybreaks
\renewcommand{\baselinestretch}{1.5}
\DeclareMathOperator*{\diag}{diag}
\DeclareMathOperator*{\cov}{cov}
\DeclareMathOperator*{\rank}{rank}
\DeclareMathOperator*{\var}{var}
\DeclareMathOperator*{\tr}{tr}
\DeclareMathOperator*{\veco}{vec}
\DeclareMathOperator*{\uniform}{\mathcal{U}niform}
\DeclareMathOperator*{\argmin}{arg\ min}
\DeclareMathOperator*{\argmax}{arg\ max}
\DeclareMathOperator*{\N}{N}
\DeclareMathOperator*{\gm}{Gamma}
\DeclareSymbolFont{extraup}{U}{zavm}{m}{n}
\DeclareMathSymbol{\varheart}{\mathalpha}{extraup}{86}
\DeclareMathSymbol{\vardiamond}{\mathalpha}{extraup}{87}
\bibliographystyle{plainnat}
\newcommand*{\doi}[1]{\href{http://dx.doi.org/#1}{doi: #1}}
\newcommand{\bs}[1]{\boldsymbol{#1}}


\begin{document}
\singlespacing
\title{Paired factor analysis (PFA)}
\author{Gao Wang, Kushal Dey and Matthew Stephens}
\maketitle

\section{The PFA model}

\subsection{Data and latent variables}
Let $D_{nj}$ be the data corresponding to $n$-th sample and $j$-th feature, 
where $n$ runs from $1$ to $N$ and $j$ runs from $1$ to $J$.  
Suppose these data come from a graph with $K$ nodes (factors) and $E$ edges. In the
PFA set up $E=\frac{K(K-1)}{2}$.

Let us define latent variables $Z$ and $\Lambda$. $Z_{n}$ is a $(E+K)
\times 1$ binary vector. We use $Z_{n,k_1,k_2}$ to indicate samples in between
nodes $k_1$ and $k_2$, and $Z_{n,l}$ to indicate samples near node $l$:
$$ Pr \left [ Z_{n, k_1, k_2} = 1 \right ] = \pi_{k_1,k_2} \hspace{1 cm} k_1 < k_2$$
$$ Pr \left [ Z_{n, l} = 1 \right ] = \pi_{l} \hspace{1 cm} l=1,2, \cdots, K$$
with the constraint that
$$ \sum_{k_1 < k_2}^E \pi_{k_1, k_2} + \sum_{l=1}^{K} \pi_{l} = 1 $$
Likelihood of $Z$ is
$$  Pr (Z |  \pi )  = \prod_{n=1}^{N} \prod_{k_1 < k_2}^E \pi_{k_1,k_2}^{z_{n,k_1,k_2}} \prod_{l=1}^{K} \pi_{l}^{z_{n,l}} $$

$\Lambda_{n}$ is a $Q \times 1$ binary vector, where $Q$ is the cardinality of
the set of coordinates for positions in between nodes that samples are fitted
to. For computational convenience we assume that these coordinates take a finite 
set of values between $0$ and $1$, say $1/100, 2/100, \cdots, 99/100, 1$.

We have for density of $\Lambda$,
$$ Pr \left [ \Lambda_{n, q} = 1 \right ] = \frac{1}{Q} $$


\subsection{The model likelihood}
For data on the graph given a pair of nodes $(k_1, k_2)$ and position $q$ on the
edge we assume a simple mixture model 

\begin{align}
 E \left [ D_{nj} | Z_{n, k_1, k_2} = 1, \Lambda_{n,q}=1, F \right] &= q
  F_{k_1,j} + (1-q) F_{k_2,j} \hspace{1 cm} k_1<k_2 \nonumber \\
 E \left [ D_{nj} | Z_{n,l} = 1, F \right] &= F_{l,j} \hspace{1 cm} k_1 = k_2 = l 
\end{align}

Where $F$ is a $K\times J$ matrix of factors. 
Then we can marginalize over $Z$ and $\Lambda$, assuming as a first pass 
that the data is Gaussian in its distribution,

\begin{multline}
 $$ Pr \left [ D_{n} | \pi, \delta, F, s^2_{j=1,2,\cdots,J} \right ] = \sum_{k_1 < k_2}
 \pi_{k_1, k_2} Pr \left [ D_{n} | Z_{n, k_1, k_2}=1, \delta, F, s^2_{j=1,2,\cdots,J} \right ]  \\
   \qquad +  \sum_{l} \pi_{l} Pr \left [ D_{n} | Z_{n, l}=1,  F, s^2_{j=1,2,\cdots,J} \right ]$$
 \end{multline}

where 

$$ Pr \left [ D_{n} | Z_{n, k_1, k_2}=1, \delta, F, s^2_{j=1,2,\cdots,J} \right ] = \sum_{q} \frac{1}{Q} Pr \left [D_{n} | Z_{n, k_1, k_2}=1, \Lambda_{n, q}=1, F, s^2_{j=1,2,\cdots,J} \right ] $$

and $s^2_{j}$ is the residual variance of the $j$th feature. Let

$$\pi_{k_1, k_2, q} = \pi_{k_1, k_2}\times\frac{1}{Q} \hspace{1 cm} k_1 < k_2$$

The overall likelihood 

\begin{align}
L(\pi, F) &= \prod_{n=1}^{N} Pr \left [ D_{n} | \pi, F, s^2_{j=1,2,\cdots,J}
\right ] \nonumber \\
 &= \prod_{n=1}^{N} \left [ \sum_{k_1 < k_2} \sum_{q=1}^{Q} \left [\pi_{k_1,k_2,q} \times \prod_{j=1}^{J} N \left (D_{nj}; q F_{k_1,j} + (1-q) F_{k_2, j}, s^2_{j} \right) \right ] + \right.\nonumber \\
 & \left. \sum_{l} \left [ \pi_{l} \times \prod_{j=1}^{J} N \left (D_{nj}; F_{l,j} , s^2_{j} \right) \right ] \right]
\end{align}

And the log likelihood

\begin{multline}
\ln {L (\pi, F)} = \sum_{n=1}^{N} \ln \left (\sum_{k_1 < k_2} \sum_{q} \left [ \pi_{k_1,k_2, q} \times \prod_{j=1}^{J} N \left (D_{nj}; q F_{k_1,j} + (1-q) F_{k_2, j}, s^2_{j} \right) \right ] + \right . \\
\left. \qquad \sum_{l=1}^{K} \pi_{l}  \times \prod_{j=1}^{J} N \left (D_{nj}; F_{l,j}, s^2_{j} \right)\right )
\end{multline}

This is the quantity we want to maximize. 

\section{EM algorithm}
\subsection{E step}

Suppose we have run upto $m$ iterations. For the $(m+1)$th iteration, we have 

\begin{eqnarray} \nonumber
\delta^{(m+1)}_{n, k_1, k_2} &=& Pr \left [ Z_{n, k_1, k_2} = 1 | \pi^{(m)}, F^{(m)}, s^{(m)}_{j=1,2,\cdots,J}, D_{n} \right ] \\ \nonumber
 &\propto& Pr \left [ Z_{n, k_1, k_2} = 1 \right] \sum_{q=1}^{Q} Pr \left [ \lambda_{n,q} = 1 \right] Pr \left [ D_{n} | \pi^{(m)}, F^{(m)}, s^{(m)}_{j=1,2,\cdots,J}, Z_{n, k_1, k_2}= 1, \lambda_{n, q}=1 \right] \\ \nonumber
 &\propto& \pi^{(m)}_{k_1,k_2} \sum_{q=1}^{Q} \left [ \prod_{j} N \left (D_{nj} | qF^{(m)}_{k_1,j} + (1-q)F^{(m)}_{k_2,j}, {s_j^{(m)}}^2 \right) \right ] \\ \nonumber
\end{eqnarray}

\begin{eqnarray} \nonumber
\delta^{(m+1)}_{n, l}  &=& Pr \left [ Z_{n, l} = 1  |  \pi^{(m)}, F^{(m)}, s^{(m)}_{j=1,2,\cdots,J}, D_{n} \right ] \\ \nonumber 
& \propto & Pr \left [ Z_{n, l} = 1 \right] Pr \left [ D_{n} | \pi^{(m)}, F^{(m)}, s^{(m)}_{j=1,2,\cdots,J}, Z_{n, l}= 1 \right] \\ \nonumber
& \propto & \pi^{(m)}_{l}  \prod_{j} N \left (D_{nj} | F^{(m)}_{l,j} , {s_j^{(m)}}^2 \right) \\  \nonumber
\end{eqnarray}


where ${s_j^{(m)}}^2$ is the residual variance of feature $j$.

We normalize $\delta$ so that 

$$ \sum_{k_1 < k_2} \delta^{(m+1)}_{n, k_1, k_2}  + \sum_{l=1}^{K} \delta^{(m+1)}_{n, l}= 1 \hspace{1 cm} \forall n $$

We define 

$$ \pi^{(m+1)}_{k_1, k_2} = \frac{1}{N}\sum_{n=1}^{N} \delta^{(m+1)}_{n, k_1, k_2} $$

$$ \pi^{(m+1)}_{l} = \frac{1}{N}\sum_{n=1}^{N} \delta^{(m+1)}_{n, l} $$

We have therefore updated $\pi^{(m)}_{k_1, k_2}$ to $\pi^{(m+1)}_{k_1, k_2}$.

\subsubsection{Variational inference}

Now we introduce prior on parameter $\pi$ for the latent variable $Z$. The
motivation is to induce sparsity on factor pairs we identify from the model.  
Here we estimate their joint variational distribution:

$$ q(Z, \Lambda, \pi) = q(Z) q(\pi) q(\Lambda) $$

We assume up front that $q^{\star}(\Lambda) = \prod_{n=1}^{N} \left [
  \frac{1}{Q} \right ]^{\Lambda_{nq}} $. For $\pi$ we use a Dirichlet prior as follows

$$ Pr (\pi | \alpha_{0}) = C (\alpha_0) \prod_{k_1 < k_2}^E \pi_{k_1, k_2}^{\alpha_0 -1} \prod_{l=1}^{L} \pi_{l}^{\alpha_0 -1}$$

\begin{eqnarray}
\ln q^{\star} (Z)  & = & E_{\pi, \Lambda} \left [ \ln p(\pi|\alpha_0) + \ln p(\Lambda) + \ln p(Z | \pi) + \ln p(D | Z, \Lambda, F, s_{j=1,2,\cdots,J}) \right ] \\ \nonumber
  & = & E_{\pi, \Lambda} \left [ \ln p(Z | \pi) + \ln p(D | Z, \Lambda, F, s_{j=1,2,\cdots,J}) \right] + constant \\\nonumber
  & = & \sum_{n=1}^{N} \sum_{k_1 < k_2}  z_{n, k_1, k_2} E_{\pi} \left [ \ln (\pi_{k_1, k_2}) \right ] \\ \nonumber
  &&  + \sum_{n=1}^{N} \sum_{k_1 < k_2}  z_{n, k_1, k_2} \sum_{q=1}^{Q} \frac{1}{Q} \left [ - \sum_{j=1}^{J} \ln (s_j) - \frac{J}{2} \ln (2 \pi) - \sum_{j=1}^{J} \frac{(D_{nj} - qF_{k_1,j} - (1-q)F_{k_2,j})^2}{2s^2_j} \right] \\ \nonumber
  &&  +  \sum_{n=1}^{N} \sum_{l}  z_{n, l} \left [ - \sum_{j=1}^{J} \ln (s_j) - \frac{J}{2} \ln (2 \pi) - \sum_{j=1}^{J} \frac{(D_{nj} - F_{l,j} )^2}{2s^2_j} \right] \\ \nonumber 
  && +  \sum_{n=1}^{N} \sum_{l=1}^{K} z_{n, l} E_{\pi} \left [ \ln (\pi_{l}) \right ] \\ \nonumber
\end{eqnarray}

\begin{eqnarray}
\ln q^{\star} (Z, \Lambda)  & = & E_{\pi} \left [ \ln p(\pi|\alpha_0) +  \ln p(\Lambda ) + \ln p(Z | \pi) + \ln p(D | Z, \Lambda, F, s_{j=1,2,\cdots,J}) \right ] \\ \nonumber
  & = & E_{\pi, \nu} \left [ \ln p(Z | \pi) + \ln p(\Lambda ) + \ln p(D | Z, \Lambda, F, s_{j=1,2,\cdots,J}) \right] + constant \\\nonumber
  & = & \sum_{n=1}^{N} \sum_{k_1 < k_2} z_{n, k_1, k_2}  E_{\pi} \left [ \ln (\pi_{k_1, k_2}) \right ]  + +  \sum_{n=1}^{N} \sum_{l=1}^{K} z_{n, l} E_{\pi} \left [ \ln (\pi_{l}) \right ] \\ \nonumber
  &&  + \sum_{n=1}^{N} \sum_{k_1 < k_2} z_{n, k_1, k_2} \left [ - \sum_{j=1}^{J} \ln (s_j) - \frac{J}{2} \ln (2 \pi) - \sum_{j=1}^{J} \frac{(D_{nj} - qF_{k_1,j} - (1-q)F_{k_2,j})^2}{2s^2_j} \right] \\ \nonumber
  &&  +  \sum_{n=1}^{N} \sum_{l}  z_{n, l} \left [ - \sum_{j=1}^{J} \ln (s_j) - \frac{J}{2} \ln (2 \pi) - \sum_{j=1}^{J} \frac{(D_{nj} - F_{l,j} )^2}{2s^2_j} \right] \\ \nonumber 
\end{eqnarray}

From here one can get 

$$ q^{\star}(Z) \propto \prod_{n=1}^{N} \left[\prod_{k_1 < k_2} \delta_{n, k_1, k_2}^{Z_{n, k_1, k_2}}  \prod_{l=1}^{K} \delta_{n, l}^{Z_{n, l}} \right]$$

$$ q^{\star} (\pi) \propto  \prod_{k_1 < k_2} \pi_{k_1, k_2}^{a_{k_1,k_2} -1} \prod_{l=1}^{L} \pi_{l}^{a_{l} -1}$$

%Similarly the prior distribution for $\nu$ is 
%
%$$ q^{\star}(\nu)  \propto    \prod_{q=1}^{Q} \nu_{q}^{b_{q} -1 }  $$
%
%
then 

\begin{multline}
 $$ \delta_{n, k_1, k_2} \propto exp \left (  E_{\pi} \left [ \ln (\pi_{k_1,k_2}) \right]  \right. \\
 \left . +  \sum_{q=1}^{Q} \frac{1}{Q} \left [ - \sum_{j=1}^{J} \ln (s_j)  - \frac{J}{2} \ln (2 \pi) - \sum_{j=1}^{J} \frac{(D_{nj} - qF_{k_1,j} - (1-q)F_{k_2,j})^2}{2s^2_j} \right] \right ) 
\end{multline}

\begin{multline}
  \delta_{n, k_1, k_2} \propto exp \left ( \psi({a_{k_1, k_2}}) - \psi(\sum_{l} a_{l} + \sum_{k_1 < k_2} a_{k_1, k_2})   \right . \\
  \left . + \sum_{q=1}^{Q} \frac{1}{Q} \left [ - \sum_{j=1}^{J} \ln (s_j)  - \frac{J}{2} \ln (2 \pi) -  \sum_{j=1}^{J} \frac{(D_{nj} - qF_{k_1,j} - (1-q)F_{k_2,j})^2}{2s^2_j} \right] \right ) 
\end{multline}

\begin{multline}
\delta_{n, l} \propto exp \left (  E_{\pi} \left [ \ln (\pi_{l}) \right] +  \left [ - \sum_{j=1}^{J} \ln (s_j)  - \frac{J}{2} \ln (2 \pi) - \sum_{j=1}^{J} \frac{(D_{nj} - F_{l,j})^2}{2s^2_j} \right] \right ) 
\end{multline}

\begin{multline}
  \delta_{n, l} \propto exp \left ( \psi (a_{l}) - \psi(\sum_{l} a_{l} + \sum_{k_1 < k_2} a_{k_1, k_2}) + \left [ - \sum_{j=1}^{J} \ln (s_j)  - \frac{J}{2} \ln (2 \pi) -  \sum_{j=1}^{J} \frac{(D_{nj} - F_{l,j})^2}{2s^2_j} \right] \right ) 
\end{multline}

We can also derive variational distributions similarly for $\pi$.

\begin{eqnarray} \nonumber
\ln q^{\star} (\pi) &= & E_{\Lambda, Z} \left [ \ln p(\pi|\alpha_0)  + \ln p(Z | \pi) + \ln p(D | Z, \Lambda, F, s_{j=1,2,\cdots,J}) \right ] \\ \nonumber
  & = & E_{Z} \left [ \ln p(Z | \pi) \right] + \ln p(\pi | \alpha_0) + constant \\ \nonumber
  & = & \sum_{n=1}^{N}\sum_{k_1 < k_2} E(z_{n, k_1, k_2}) \ln \pi_{k_1,k_2} + \sum_{n=1}^{N}\sum_{l=1}^{K} E(z_{n, l}) \ln \pi_{l} +  (\alpha_0 -1) \sum_{k_1 < k_2} \ln \pi_{k_1,k_2} \\ \nonumber
  & = & \sum_{k_1 < k_2} \left [ \sum_{n=1}^{N} \delta_{n, k_1, k_2} + (\alpha_0 -1) \right] \ln \pi_{k_1, k_2} + \sum_{l=1}^{K} \left [ \sum_{n=1}^{N} \delta_{n, l} + (\alpha_0 -1) \right] \ln \pi_{l}  \\ \nonumber
\end{eqnarray}

We define 

$$ a_{k_1, k_2} = \alpha_0 + \sum_{n=1}^{N} \delta_{n, k_1, k_2} $$
$$ a_{l} = \alpha_0 + \sum_{n=1}^{N} \delta_{n, l} $$


$$ q^{\star} (\pi) = Dir(\pi | a)  $$


We initialize $\pi$  first, then use $a_{k_1,k_2}=\alpha_0$  to begin with and estimate $\delta_{n,k_1,k_2}$. Then use the $\delta_{n,k_1,k_2}$ to update $a_{k_1,k_2}$  and proceed in this way. In this case, we do not assume independence of the $\Lambda$ and $Z$ variational distributions, so this model is more generalized.

\subsection{M step}


%We define the parameter 
%
%$$ \theta : = \left (\pi_{k_1,k_2, q}, F, s_{j=1,2,\cdots,J} \right ) $$
%
%We define the complete loglikelihood 
%
%$$ log L_{c} \left (\theta; D, Z, \lambda \right ) = log \pi_{k_1,k_2,q} + log L (D | Z, \lambda, q, F) $$
%

We define 

$$ \delta^{(m+1)}_{n, k_1, k_2,q} =  \delta^{(m+1)}_{n, k_1, k_2} \times \frac{1}{Q}  $$ 

and take the expectation of this quantity with respect to $\left [ Z, \lambda | D, \theta^{(m)} \right ]$:

\begin{eqnarray}
 Q (\theta | \theta^{(m)}) \propto - \sum_{n=1}^{N} \sum_{k_1 < k_2} \sum_{q=1}^{Q} \delta^{(m+1)}_{n, k_1, k_2, q}  \sum_{j} \left [ log s^{(m+1)}_{j} + \frac{(D_{nj} - q F_{k_1,j} - (1-q) F_{k_2,j})^2}{2{s_j^{(m+1)}}^2} \right] \\
 -   \sum_{n=1}^{N} \sum_{l=1}^{K} \sum_{q=1}^{Q} \delta^{(m+1)}_{n, l, q}  \sum_{j} \left [ log s^{(m+1)}_{j} + \frac{(D_{nj} - F_{l,j})^2}{2{s_j^{(m+1)}}^2} \right] 
\end{eqnarray}

We try to maximize this quantity with respect to $F$, ie, we can take derivative with respect to $F$ and try to solve the resulting normal equation.

This equation, conditional on $\left [ Z, \lambda | D, \theta^{(m)} \right ]$, can be written as 

\begin{eqnarray}
 D_{N \times J} = L_{N \times K} F_{K \times J} + E_{N \times J}
\end{eqnarray}

where 

$$ e_{nj} \sim N(0, s^2_{j}) $$

We define 

$$ D^{'}_{nj} : = \frac{D_{nj}}{s_{j}} $$

If we consider finding the factors on a feature-by-feature basis, we do not need to worry about $s_j$.

\begin{align*}
L_{nk} =
\begin{cases}
    q~\text{or}~(1-q) & \lambda_{n}=q  \; \; Z_{n,k,*}=1 \; \text{or} \;\; Z_{n, *, k} = 1 \\
    1 & Z_{n,k}=1 \\
    0 & \text{o.w.}
\end{cases}
\end{align*}

We have 

$$ E_{ Z, \lambda | D, \theta^{(m)}} \left [ L_{nk} \right ] = \sum_{q}  \sum_{k_2 > k} q \delta^{(m+1)}_{n, k, k_2, q}  + \sum_{q}  \sum_{k_1 < k} (1-q) \delta^{(m+1)}_{n, k_1, k, q} +  \delta^{(m+1)}_{n, k} $$

$$ E_{ Z, \lambda | D, \theta^{(m)}} \left [ L^2_{nk} \right ] = \sum_{q}  \sum_{k_2 > k} q^2 \delta^{(m+1)}_{n,k,k_2, q}  + \sum_{q}  \sum_{k_1 < k} (1-q)^2 \delta^{(m+1)}_{n,k_1,k,q} +  \delta^{(m+1)}_{n, 0, 0, k} $$

Also for any $k \neq l$,

$$ E_{ Z, \lambda | D, \theta^{(m)}} \left [ L_{nk}L_{nl} \right ] =
\sum_{q} q(1-q) \delta^{(m+1)}_{n,k,l,q} $$

We use these to solve for the equation

$$ \left [ E_{ Z, \lambda | D, \theta^{(m)}} \left( L^{T}L \right ) \right ] F \approx \left [ E_{ Z, \lambda | D, \theta^{(m)}} (L) \right] ^{T} D $$

The solution therefore is 

$$ F \approx \left [ E_{ Z, \lambda | D, \theta^{(m)}} \left( L^{T}L \right ) \right]^{-1} \left [ E_{ Z, \lambda | D, \theta^{(m)}} (L) \right]^{T} D $$

For $W = L^{T}L$

$$ W_{kl} = \sum_{n} L_{kn}L_{nl} $$

$$ E_{ Z, \lambda | D, \theta^{(m)}} \left ( W_{kl} \right ) = \sum_{n}  E_{ Z, \lambda | D, \theta^{(m)}} \left ( L_{nk}L_{nl} \right) $$

We use the definition of $E_{ Z, \lambda | D, \theta^{(m)}} \left [ L^2_{nk} \right ]$ 
and $E_{ Z, \lambda | D, \theta^{(m)}} \left [ L_{nk}L_{nl} \right ]$ 
from above to solve $F$. 

In the same way as we computed $F$ by solving for the normal equation obtained from taking derivative of the function $Q (\theta | \theta^{(m)})$, we take derivative of the latter with respect to $s^2_{j}$ to obtain EM updates of the residual variance terms. Taking the derivative, we obtain the estimate as 

\begin{multline}
\widehat{s_{j}^{(m+1)}}^2 = \frac{1}{N}\sum_{n=1}^{N} \sum_{k_1 < k_2} \sum_{q} \delta^{(m+1)}_{n, k_1, k_2, q} (D_{nj} - q F_{k_1,j} - (1-q) F_{k_2,j})^2 \\
+  \frac{1}{N}\sum_{n=1}^{N} \sum_{l=1}^{K} \delta^{(m+1)}_{n, l, q} (D_{nj} - F_{l,j} )^2
\end{multline}

where the $F$ are the estimated values of the factors from the previous step.

We then continue this procedure described above for multiple iterations.

\end{document}